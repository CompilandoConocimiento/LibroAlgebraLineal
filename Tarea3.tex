% ****************************************************************************************
% ************************          TAREA 2           ************************************
% ****************************************************************************************



% =======================================================
% =======         HEADER FOR DOCUMENT        ============
% =======================================================
    
    % *********   HEADERS AND FOOTERS ********
    \def\ProjectAuthorLink{https://github.com/SoyOscarRH}           %Just to keep it in line
    \def\ProjectNameLink{\ProjectAuthorLink/Proyect}                %Link to Proyect

    % *********   DOCUMENT ITSELF   **************
    \documentclass[12pt, fleqn]{article}                             %Type of docuemtn and size of font and left eq
    \usepackage[spanish]{babel}                                     %Please use spanish
    \usepackage[utf8]{inputenc}                                     %Please use spanish - UFT
    \usepackage[margin = 1.2in]{geometry}                           %Margins and Geometry pacakge
    \usepackage{ifthen}                                             %Allow simple programming
    \usepackage{hyperref}                                           %Create MetaData for a PDF and LINKS!
    \usepackage{pdfpages}                                           %Create MetaData for a PDF and LINKS!
    \hypersetup{pageanchor = false}                                 %Solve 'double page 1' warnings in build
    \setlength{\parindent}{0pt}                                     %Eliminate ugly indentation
    \author{Oscar Andrés Rosas}                                     %Who I am

    % *********   LANGUAJE    *****************
    \usepackage[T1]{fontenc}                                        %Please use spanish
    \usepackage{textcmds}                                           %Allow us to use quoutes
    \usepackage{changepage}                                         %Allow us to use identate paragraphs
    \usepackage{anyfontsize}                                        %All the sizes

    % *********   MATH AND HIS STYLE  *********
    \usepackage{ntheorem, amsmath, amssymb, amsfonts}               %All fucking math, I want all!
    \usepackage{mathrsfs, mathtools, empheq}                        %All fucking math, I want all!
    \usepackage{cancel}                                             %Negate symbol
    \usepackage{centernot}                                          %Allow me to negate a symbol
    \decimalpoint                                                   %Use decimal point

    % *********   GRAPHICS AND IMAGES *********
    \usepackage{graphicx}                                           %Allow to create graphics
    \usepackage{float}                                              %For images
    \usepackage{wrapfig}                                            %Allow to create images
    \graphicspath{ {Graphics/} }                                    %Where are the images :D

    % *********   LISTS AND TABLES ***********
    \usepackage{listings, listingsutf8}                             %We will be using code here
    \usepackage[inline]{enumitem}                                   %We will need to enumarate
    \usepackage{tasks}                                              %Horizontal lists
    \usepackage{longtable}                                          %Lets make tables awesome
    \usepackage{booktabs}                                           %Lets make tables awesome
    \usepackage{tabularx}                                           %Lets make tables awesome
    \usepackage{multirow}                                           %Lets make tables awesome
    \usepackage{multicol}                                           %Create multicolumns

    % *********   HEADERS AND FOOTERS ********
    \usepackage{fancyhdr}                                           %Lets make awesome headers/footers
    \pagestyle{fancy}                                               %Lets make awesome headers/footers
    \setlength{\headheight}{16pt}                                   %Top line
    \setlength{\parskip}{0.5em}                                     %Top line
    \renewcommand{\footrulewidth}{0.5pt}                            %Bottom line

    \lhead{                                                         %Left Header
        \hyperlink{section.\arabic{section}}                        %Make a link to the current chapter
        {\normalsize{\textsc{\nouppercase{\leftmark}}}}             %And fot it put the name
    }

    \rhead{                                                         %Right Header
        \hyperlink{section.\arabic{section}.\arabic{subsection}}    %Make a link to the current chapter
            {\footnotesize{\textsc{\nouppercase{\rightmark}}}}      %And fot it put the name
    }

    \rfoot{\textsc{\small{\hyperref[sec:Index]{Ve al Índice}}}}     %This will always be a footer  

    \fancyfoot[L]{                                                  %Algoritm for a changing footer
        \ifthenelse{\isodd{\value{page}}}                           %IF ODD PAGE:
            {\href{https://compilandoconocimiento.com/nosotros/}    %DO THIS:
                {\footnotesize                                      %Send the page
                    {\textsc{Oscar Andrés Rosas}}}}                 %Send the page
            {\href{https://compilandoconocimiento.com}              %ELSE DO THIS: 
                {\footnotesize                                      %Send the author
                    {\textsc{Algebra Lineal 1}}}}                   %Send the author
    }
    
    
    
% =======================================================
% ===================   COMMANDS    =====================
% =======================================================

    % =========================================
    % =======   NEW ENVIRONMENTS   ============
    % =========================================
    \newenvironment{Indentation}[1][0.75em]                         %Use: \begin{Inde...}[Num]...\end{Inde...}
        {\begin{adjustwidth}{#1}{}}                                 %If you dont put nothing i will use 0.75 em
        {\end{adjustwidth}}                                         %This indentate a paragraph
    \newenvironment{SmallIndentation}[1][0.75em]                    %Use: The same that we upper one, just 
        {\begin{adjustwidth}{#1}{}\begin{footnotesize}}             %footnotesize size of letter by default
        {\end{footnotesize}\end{adjustwidth}}                       %that's it

    \newenvironment{MultiLineEquation}[1]                           %Use: To create MultiLine equations
        {\begin{equation}\begin{alignedat}{#1}}                     %Use: \begin{Multi..}{Num. de Columnas}
        {\end{alignedat}\end{equation}}                             %And.. that's it!
    \newenvironment{MultiLineEquation*}[1]                          %Use: To create MultiLine equations
        {\begin{equation*}\begin{alignedat}{#1}}                    %Use: \begin{Multi..}{Num. de Columnas}
        {\end{alignedat}\end{equation*}}                            %And.. that's it!
    

    % =========================================
    % == GENERAL TEXT & SYMBOLS ENVIRONMENTS ==
    % =========================================
    
    % =====  TEXT  ======================
    \newcommand \Quote {\qq}                                        %Use: \Quote to use quotes
    \newcommand \Over {\overline}                                   %Use: \Bar to use just for short
    \newcommand \ForceNewLine {$\Space$\\}                          %Use it in theorems for example

    % =====  SPACES  ====================
    \DeclareMathOperator \Space {\quad}                             %Use: \Space for a cool mega space
    \DeclareMathOperator \MegaSpace {\quad \quad}                   %Use: \MegaSpace for a cool mega mega space
    \DeclareMathOperator \MiniSpace {\;}                            %Use: \Space for a cool mini space
    
    % =====  MATH TEXT  =================
    \newcommand \Such {\MiniSpace | \MiniSpace}                     %Use: \Such like in sets
    \newcommand \Also {\MiniSpace \text{y} \MiniSpace}              %Use: \Also so it's look cool
    \newcommand \Remember[1]{\Space\text{\scriptsize{#1}}}          %Use: \Remember so it's look cool
    
    % =====  THEOREMS  ==================
    \newtheorem{Theorem}{Teorema}[section]                          %Use: \begin{Theorem}[Name]\label{Nombre}...
    \newtheorem{Corollary}{Colorario}[Theorem]                      %Use: \begin{Corollary}[Name]\label{Nombre}...
    \newtheorem{Lemma}[Theorem]{Lemma}                              %Use: \begin{Lemma}[Name]\label{Nombre}...
    \newtheorem{Definition}{Definición}[section]                    %Use: \begin{Definition}[Name]\label{Nombre}...
    \theoremstyle{break}                                            %THEOREMS START 1 SPACE AFTER

    % =====  LOGIC  =====================
    \newcommand \lIff    {\leftrightarrow}                          %Use: \lIff for logic iff
    \newcommand \lEqual  {\MiniSpace \Leftrightarrow \MiniSpace}    %Use: \lEqual for a logic double arrow
    \newcommand \lInfire {\MiniSpace \Rightarrow \MiniSpace}        %Use: \lInfire for a logic infire
    \newcommand \lLongTo {\longrightarrow}                          %Use: \lLongTo for a long arrow

    % =====  FAMOUS SETS  ===============
    \DeclareMathOperator \Naturals     {\mathbb{N}}                 %Use: \Naturals por Notation
    \DeclareMathOperator \Primes       {\mathbb{P}}                 %Use: \Primes por Notation
    \DeclareMathOperator \Integers     {\mathbb{Z}}                 %Use: \Integers por Notation
    \DeclareMathOperator \Racionals    {\mathbb{Q}}                 %Use: \Racionals por Notation
    \DeclareMathOperator \Reals        {\mathbb{R}}                 %Use: \Reals por Notation
    \DeclareMathOperator \Complexs     {\mathbb{C}}                 %Use: \Complex por Notation
    \DeclareMathOperator \GenericField {\mathbb{F}}                 %Use: \GenericField por Notation
    \DeclareMathOperator \VectorSet    {\mathbb{V}}                 %Use: \VectorSet por Notation
    \DeclareMathOperator \SubVectorSet {\mathbb{W}}                 %Use: \SubVectorSet por Notation
    \DeclareMathOperator \Polynomials  {\mathbb{P}}                 %Use: \Polynomials por Notation
    \DeclareMathOperator \VectorSpace  {\VectorSet_{\GenericField}} %Use: \VectorSpace por Notation
    \DeclareMathOperator \LinealTransformation {\mathcal{T}}        %Use: \LinealTransformation for a cool T
    \DeclareMathOperator \LinTrans {\mathcal{T}}                    %Use: \LinTrans for a cool T
    \DeclareMathOperator \Laplace {\mathcal{L}}                     %Use: \LinTrans for a cool T


    % =====  CONTAINERS   ===============
    \newcommand{\Set}[1]    {\left\{ \; #1 \; \right\}}             %Use: \Set {Info} for INTELLIGENT space 
    \newcommand{\bigSet}[1] {\big\{  \; #1 \; \big\}}               %Use: \bigSet  {Info} for space 
    \newcommand{\BigSet}[1] {\Big\{  \; #1 \; \Big\}}               %Use: \BigSet  {Info} for space 
    \newcommand{\biggSet}[1]{\bigg\{ \; #1 \; \bigg\}}              %Use: \biggSet {Info} for space 
    \newcommand{\BiggSet}[1]{\Bigg\{ \; #1 \; \Bigg\}}              %Use: \BiggSet {Info} for space 
    
    \newcommand{\Brackets}[1]    {\left[ #1 \right]}                %Use: \Brackets {Info} for INTELLIGENT space
    \newcommand{\bigBrackets}[1] {\big[ \; #1 \; \big]}             %Use: \bigBrackets  {Info} for space 
    \newcommand{\BigBrackets}[1] {\Big[ \; #1 \; \Big]}             %Use: \BigBrackets  {Info} for space 
    \newcommand{\biggBrackets}[1]{\bigg[ \; #1 \; \bigg]}           %Use: \biggBrackets {Info} for space 
    \newcommand{\BiggBrackets}[1]{\Bigg[ \; #1 \; \Bigg]}           %Use: \BiggBrackets {Info} for space 
    
    \newcommand{\Wrap}[1]    {\left( #1 \right)}                    %Use: \Wrap {Info} for INTELLIGENT space
    \newcommand{\bigWrap}[1] {\big( \; #1 \; \big)}                 %Use: \bigBrackets  {Info} for space 
    \newcommand{\BigWrap}[1] {\Big( \; #1 \; \Big)}                 %Use: \BigBrackets  {Info} for space 
    \newcommand{\biggWrap}[1]{\bigg( \; #1 \; \bigg)}               %Use: \biggBrackets {Info} for space 
    \newcommand{\BiggWrap}[1]{\Bigg( \; #1 \; \Bigg)}               %Use: \BiggBrackets {Info} for space 
    
    \newcommand{\Generate}[1]{\left\langle #1 \right\rangle}        %Use: \Wrap {Info} for INTELLIGENT space

    % =====  BETTERS MATH COMMANDS   =====
    \newcommand{\pfrac}[2]{\Wrap{\dfrac{#1}{#2}}}                   %Use: Put fractions in parentesis

    % =========================================
    % ====   LINEAL ALGEBRA & VECTORS    ======
    % =========================================

    % ===== UNIT VECTORS  ================
    \newcommand{\hati} {\hat{\imath}}                               %Use: \hati for unit vector    
    \newcommand{\hatj} {\hat{\jmath}}                               %Use: \hatj for unit vector    
    \newcommand{\hatk} {\hat{k}}                                    %Use: \hatk for unit vector

    % ===== FN LINEAL TRANSFORMATION  ====
    \newcommand{\FnLinTrans}[1]{\mathcal{T}\Wrap{#1}}               %Use: \FnLinTrans for a cool T
    \newcommand{\VecLinTrans}[1]{\mathcal{T}\pVector{#1}}           %Use: \LinTrans for a cool T
    \newcommand{\FnLinealTransformation}[1]{\mathcal{T}\Wrap{#1}}   %Use: \FnLinealTransformation

    % ===== MAGNITUDE  ===================
    \newcommand{\abs}[1]{\left\lvert #1 \right\lvert}               %Use: \abs{expression} for |x|
    \newcommand{\Abs}[1]{\left\lVert #1 \right\lVert}               %Use: \Abs{expression} for ||x||
    \newcommand{\Mag}[1]{\left| #1 \right|}                         %Use: \Mag {Info} 
    
    \newcommand{\bVec}[1]{\mathbf{#1}}                              %Use for bold type of vector
    \newcommand{\lVec}[1]{\overrightarrow{#1}}                      %Use for a long arrow over a vector
    \newcommand{\uVec}[1]{\mathbf{\hat{#1}}}                        %Use: Unitary Vector Example: $\uVec{i}

    % ===== ALL FOR DOT PRODUCT  =========
    \makeatletter                                                   %WTF! IS THIS
    \newcommand*\dotP{\mathpalette\dotP@{.5}}                       %Use: \dotP for dot product
    \newcommand*\dotP@[2] {\mathbin {                               %WTF! IS THIS            
        \vcenter{\hbox{\scalebox{#2}{$\m@th#1\bullet$}}}}           %WTF! IS THIS
    }                                                               %WTF! IS THIS
    \makeatother                                                    %WTF! IS THIS

    % === WRAPPERS FOR COLUMN VECTOR ===
    \newcommand{\pVector}[1]                                        %Use: \pVector {Matrix Notation} use parentesis
        { \ensuremath{\begin{pmatrix}#1\end{pmatrix}} }             %Example: \pVector{a\\b\\c} or \pVector{a&b&c} 
    \newcommand{\lVector}[1]                                        %Use: \lVector {Matrix Notation} use a abs 
        { \ensuremath{\begin{vmatrix}#1\end{vmatrix}} }             %Example: \lVector{a\\b\\c} or \lVector{a&b&c} 
    \newcommand{\bVector}[1]                                        %Use: \bVector {Matrix Notation} use a brackets 
        { \ensuremath{\begin{bmatrix}#1\end{bmatrix}} }             %Example: \bVector{a\\b\\c} or \bVector{a&b&c} 
    \newcommand{\Vector}[1]                                         %Use: \Vector {Matrix Notation} no parentesis
        { \ensuremath{\begin{matrix}#1\end{matrix}} }               %Example: \Vector{a\\b\\c} or \Vector{a&b&c}

    % === MAKE MATRIX BETTER  =========
    \makeatletter                                                   %Example: \begin{matrix}[cc|c]
    \renewcommand*\env@matrix[1][*\c@MaxMatrixCols c] {             %WTF! IS THIS
        \hskip -\arraycolsep                                        %WTF! IS THIS
        \let\@ifnextchar\new@ifnextchar                             %WTF! IS THIS
        \array{#1}                                                  %WTF! IS THIS
    }                                                               %WTF! IS THIS
    \makeatother                                                    %WTF! IS THIS

    % =========================================
    % =======   FAMOUS FUNCTIONS   ============
    % =========================================

    % == TRIGONOMETRIC FUNCTIONS  ====
    \newcommand{\Cos}[1] {\cos\Wrap{#1}}                            %Simple wrappers
    \newcommand{\Sin}[1] {\sin\Wrap{#1}}                            %Simple wrappers
    \newcommand{\Tan}[1] {tan\Wrap{#1}}                             %Simple wrappers
    
    \newcommand{\Sec}[1] {sec\Wrap{#1}}                             %Simple wrappers
    \newcommand{\Csc}[1] {csc\Wrap{#1}}                             %Simple wrappers
    \newcommand{\Cot}[1] {cot\Wrap{#1}}                             %Simple wrappers

    % === COMPLEX ANALYSIS TRIG ======
    \newcommand \Cis[1]  {\Cos{#1} + i \Sin{#1}}                    %Use: \Cis for cos(x) + i sin(x)
    \newcommand \pCis[1] {\Wrap{\Cis{#1}}}                          %Use: \pCis for the same with parantesis
    \newcommand \bCis[1] {\Brackets{\Cis{#1}}}                      %Use: \bCis for the same with Brackets


    % =========================================
    % ===========     CALCULUS     ============
    % =========================================

    % ====== TRANSFORMS =============
    \newcommand{\FourierT}[1]{\mathscr{F} \left\{ #1 \right\} }     %Use: \FourierT {Funtion}
    \newcommand{\InvFourierT}[1]{\mathscr{F}^{-1}\left\{#1\right\}} %Use: \InvFourierT {Funtion}

    % ====== DERIVATIVES ============
    \newcommand \MiniDerivate[1][x] {\dfrac{d}{d #1}}               %Use: \MiniDerivate[var] for simple use [var]
    \newcommand \Derivate[2] {\dfrac{d \; #1}{d #2}}                %Use: \Derivate [f(x)][x]
    \newcommand \MiniUpperDerivate[2] {\dfrac{d^{#2}}{d#1^{#2}}}    %Mini Derivate High Orden Derivate -- [x][pow]
    \newcommand \UpperDerivate[3] {\dfrac{d^{#3} \; #1}{d#2^{#3}}}  %Complete High Orden Derivate -- [f(x)][x][pow]
    
    \newcommand \MiniPartial[1][x] {\dfrac{\partial}{\partial #1}}  %Use: \MiniDerivate for simple use [var]
    \newcommand \Partial[2] {\dfrac{\partial \; #1}{\partial #2}}   %Complete Partial Derivate -- [f(x)][x]
    \newcommand \MiniUpperPartial[2]                                %Mini Derivate High Orden Derivate -- [x][pow] 
        {\dfrac{\partial^{#2}}{\partial #1^{#2}}}                   %Mini Derivate High Orden Derivate
    \newcommand \UpperPartial[3]                                    %Complete High Orden Derivate -- [f(x)][x][pow]
        {\dfrac{\partial^{#3} \; #1}{\partial#2^{#3}}}              %Use: \UpperDerivate for simple use

    \DeclareMathOperator \Evaluate  {\Big|}                         %Use: \Evaluate por Notation

    % =========================================
    % ========    GENERAL STYLE     ===========
    % =========================================
    
    % =====  COLORS ==================
    \definecolor{RedMD}{HTML}{F44336}                               %Use: Color :D        
    \definecolor{Red100MD}{HTML}{FFCDD2}                            %Use: Color :D        
    \definecolor{Red200MD}{HTML}{EF9A9A}                            %Use: Color :D        
    \definecolor{Red300MD}{HTML}{E57373}                            %Use: Color :D        
    \definecolor{Red700MD}{HTML}{D32F2F}                            %Use: Color :D 

    \definecolor{PurpleMD}{HTML}{9C27B0}                            %Use: Color :D        
    \definecolor{Purple100MD}{HTML}{E1BEE7}                         %Use: Color :D        
    \definecolor{Purple200MD}{HTML}{EF9A9A}                         %Use: Color :D        
    \definecolor{Purple300MD}{HTML}{BA68C8}                         %Use: Color :D        
    \definecolor{Purple700MD}{HTML}{7B1FA2}                         %Use: Color :D 

    \definecolor{IndigoMD}{HTML}{3F51B5}                            %Use: Color :D        
    \definecolor{Indigo100MD}{HTML}{C5CAE9}                         %Use: Color :D        
    \definecolor{Indigo200MD}{HTML}{9FA8DA}                         %Use: Color :D        
    \definecolor{Indigo300MD}{HTML}{7986CB}                         %Use: Color :D        
    \definecolor{Indigo700MD}{HTML}{303F9F}                         %Use: Color :D 

    \definecolor{BlueMD}{HTML}{2196F3}                              %Use: Color :D        
    \definecolor{Blue100MD}{HTML}{BBDEFB}                           %Use: Color :D        
    \definecolor{Blue200MD}{HTML}{90CAF9}                           %Use: Color :D        
    \definecolor{Blue300MD}{HTML}{64B5F6}                           %Use: Color :D        
    \definecolor{Blue700MD}{HTML}{1976D2}                           %Use: Color :D        
    \definecolor{Blue900MD}{HTML}{0D47A1}                           %Use: Color :D  

    \definecolor{CyanMD}{HTML}{00BCD4}                              %Use: Color :D        
    \definecolor{Cyan100MD}{HTML}{B2EBF2}                           %Use: Color :D        
    \definecolor{Cyan200MD}{HTML}{80DEEA}                           %Use: Color :D        
    \definecolor{Cyan300MD}{HTML}{4DD0E1}                           %Use: Color :D        
    \definecolor{Cyan700MD}{HTML}{0097A7}                           %Use: Color :D        
    \definecolor{Cyan900MD}{HTML}{006064}                           %Use: Color :D 

    \definecolor{TealMD}{HTML}{009688}                              %Use: Color :D        
    \definecolor{Teal100MD}{HTML}{B2DFDB}                           %Use: Color :D        
    \definecolor{Teal200MD}{HTML}{80CBC4}                           %Use: Color :D        
    \definecolor{Teal300MD}{HTML}{4DB6AC}                           %Use: Color :D        
    \definecolor{Teal700MD}{HTML}{00796B}                           %Use: Color :D        
    \definecolor{Teal900MD}{HTML}{004D40}                           %Use: Color :D 

    \definecolor{GreenMD}{HTML}{4CAF50}                             %Use: Color :D        
    \definecolor{Green100MD}{HTML}{C8E6C9}                          %Use: Color :D        
    \definecolor{Green200MD}{HTML}{A5D6A7}                          %Use: Color :D        
    \definecolor{Green300MD}{HTML}{81C784}                          %Use: Color :D        
    \definecolor{Green700MD}{HTML}{388E3C}                          %Use: Color :D        
    \definecolor{Green900MD}{HTML}{1B5E20}                          %Use: Color :D

    \definecolor{AmberMD}{HTML}{FFC107}                             %Use: Color :D        
    \definecolor{Amber100MD}{HTML}{FFECB3}                          %Use: Color :D        
    \definecolor{Amber200MD}{HTML}{FFE082}                          %Use: Color :D        
    \definecolor{Amber300MD}{HTML}{FFD54F}                          %Use: Color :D        
    \definecolor{Amber700MD}{HTML}{FFA000}                          %Use: Color :D        
    \definecolor{Amber900MD}{HTML}{FF6F00}                          %Use: Color :D

    \definecolor{BlueGreyMD}{HTML}{607D8B}                          %Use: Color :D        
    \definecolor{BlueGrey100MD}{HTML}{CFD8DC}                       %Use: Color :D        
    \definecolor{BlueGrey200MD}{HTML}{B0BEC5}                       %Use: Color :D        
    \definecolor{BlueGrey300MD}{HTML}{90A4AE}                       %Use: Color :D        
    \definecolor{BlueGrey700MD}{HTML}{455A64}                       %Use: Color :D        
    \definecolor{BlueGrey900MD}{HTML}{263238}                       %Use: Color :D        

    \definecolor{DeepPurpleMD}{HTML}{673AB7}                        %Use: Color :D

    \newcommand{\Color}[2]{\textcolor{#1}{#2}}                      %Simple color environment
    \newenvironment{ColorText}[1]                                   %Use: \begin{ColorText}
        { \leavevmode\color{#1}\ignorespaces }                      %That's is!

    % =====  CODE EDITOR =============
    \lstdefinestyle{CompilandoStyle} {                              %This is Code Style
        backgroundcolor     = \color{BlueGrey900MD},                %Background Color  
        basicstyle          = \tiny\color{white},                   %Style of text
        commentstyle        = \color{BlueGrey200MD},                %Comment style
        stringstyle         = \color{Green300MD},                   %String style
        keywordstyle        = \color{Blue300MD},                    %keywords style
        numberstyle         = \tiny\color{TealMD},                  %Size of a number
        frame               = shadowbox,                            %Adds a frame around the code
        breakatwhitespace   = true,                                 %Style   
        breaklines          = true,                                 %Style   
        showstringspaces    = false,                                %Hate those spaces                  
        breaklines          = true,                                 %Style                   
        keepspaces          = true,                                 %Style                   
        numbers             = left,                                 %Style                   
        numbersep           = 10pt,                                 %Style 
        xleftmargin         = \parindent,                           %Style 
        tabsize             = 4,                                    %Style
        inputencoding       = utf8/latin1                           %Allow me to use special chars
    }
 
    \lstset{style = CompilandoStyle}                                %Use this style









% =====================================================
% ============        COVER PAGE       ================
% =====================================================
\begin{document}
\begin{titlepage}
    
    % ============ TITLE PAGE STYLE  ================
    \definecolor{TitlePageColor}{cmyk}{1,.60,0,.40}                 %Simple colors
    \definecolor{ColorSubtext}{cmyk}{1,.50,0,.10}                   %Simple colors
    \newgeometry{left=0.25\textwidth}                               %Defines an Offset
    \pagecolor{TitlePageColor}                                      %Make it this Color to page
    \color{white}                                                   %General things should be white

    % ===== MAKE SOME SPACE =========
    \vspace                                                         %Give some space
    \baselineskip                                                   %But we need this to up command

    % ============ NAME OF THE PROJECT  ============
    \makebox[0pt][l]{\rule{1.3\textwidth}{3pt}}                     %Make a cool line
    
    \href{https://compilandoconocimiento.com}                       %Link to project
    {\textbf{\textsc{\Huge Facultad de Ciencias - UNAM}}}\\[2.7cm]  %Name of project   

    % ============ NAME OF THE BOOK  ===============
    \href{\ProjectNameLink/LibroAlgebraLineal}                      %Link to Author
    {\fontsize{65}{78}\selectfont \textbf{3 Tarea-Examen}\\[0.5cm]  %Name of the book
    \textcolor{ColorSubtext}{\textsc{\Huge Algebra Lineal 1 }}}     %Name of the general theme
    
    \vfill                                                          %Fill the space
    
    % ============ NAME OF THE AUTHOR  =============
    \href{\ProjectAuthorLink}                                       %Link to Author
    {\LARGE \textsf{Oscar Andrés Rosas Hernandez}}                  %Author

    % ===== MAKE SOME SPACE =========
    \vspace                                                         %Give some space
    \baselineskip                                                   %But we need this to up command
    
    {\large \textsf{Mayo 2018}}                                     %Date
 
\end{titlepage}


% =====================================================
% ==========      RESTORE TO DOCUMENT      ============
% =====================================================
\restoregeometry                                                    %Restores the geometry
\nopagecolor                                                        %Use to restore the color to white




% =====================================================
% ========                INDICE              =========
% =====================================================
\tableofcontents{}
\label{sec:Index}

\clearpage




% ==============================================================
% =================          PROBLEMA 1       ==================
% ==============================================================
\clearpage
\section{1 Problema}
        
    Demuestra que $E$ es una matriz elemental si y solo si $E^t$ es una matriz elemental.


    % ======== DEMOSTRACION ========
    \begin{SmallIndentation}[1em]
        \textbf{Demostración}:
        
        Ok, quería hacer una demostración general, pero para ser mas claro, vayamos por casos
        \begin{itemize}
            \item Las Matrices Elementales de Tipo 1

                Estas son simetricas, veamos:
                % ======== DEMOSTRACION ========
                \begin{SmallIndentation}[1em]
                    \textbf{Demostración}:
                    
                    Por definición una matriz elemental de Tipo 1 es la matriz identidad de $n \times n$
                    donde la fila $i$ ha sido cambiada por la fila $j$.

                    Ahora, veamos como es su transpuesta, por un lado, si miramos en filas diferentes a la $i, j$
                    entonces estamos viendo a la identidad, que es solo $1$ si estamos en la diagonal principal, por
                    lo tanto es simétrica.

                    Ahora, ahora, al momento de hacer el cambio lo único que hemos hecho ha sido cambiar un $1$ en la
                    posición $i, i$ por uno en la posición $i, j$ y un 1 en la posición $j, j$ por un 1 en la posición $j, i$.

                    Esto creo que es más que obvio que es simetrico.

                \end{SmallIndentation}

            \item Las Matrices Elementales de Tipo 2

                Estas son simetricas, veamos:
                % ======== DEMOSTRACION ========
                \begin{SmallIndentation}[1em]
                    \textbf{Demostración}:
                    
                    Estas son en las que multiplicas una fila o una columna $i$ por un escalar diferene de cero, es decir son la identidad
                    excepto en la posición $i, i$, así que es mas ue obvio que sigue siendo simetrica.
                
                \end{SmallIndentation}
                    
                    

            \item
                Las Matrices Elementales de Tipo 3

                % ======== DEMOSTRACION ========
                \begin{SmallIndentation}[1em]
                    \textbf{Demostración}:
                    
                    Estas no son simetricas.

                    Recordemos, son sumarle a una fila $i$ un multiplico $k$ de otra, digamos la $j$.

                    Por lo mismo, son iguales que la identidad expecto en la posición $i, j$ que tiene $k$.

                    Ahora, cuando la transpones tenemos la identidad excepto en la posición $j, i$ que tiene $k$.
                    Es decir en la matriz resultante es una matriz elemental de tipo 3, donde se suma a la fila
                    $j$ un multiplo $k$ de la fila $i$.
                
                \end{SmallIndentation}

        \end{itemize}
    
    \end{SmallIndentation}
            

% ==============================================================
% =================          PROBLEMA 2       ==================
% ==============================================================
\clearpage
\section{2 Problema}
                            
    Dada $A \in M_{m \times n} (\GenericField)$. Prueba que si $B$ se puede obtener mediante un
    número finito de operaciones elementales aplicadas a A, entonces $B^t$
    se puede obtener mediante las mismas operaciones elementales pero
    aplicadas a $A^t$

    % ======== DEMOSTRACION ========
    \begin{SmallIndentation}[1em]
        \textbf{Demostración}:
        
        Si $B$ se puede obtener de A por operaciones elementales sobre las filas entonces
        podemos escribir que $B = EA$ donde $E$ es un producto de operaciones elementales.

        Entonces podemos escribir que $B^t = E^t A^t$, esto nos dice que podemos obtener 
        a $B^t$ a tráves de operaciones elementales sobre columnas a partir de $A^t$

        De un modo análogo para las columnas.
    
    \end{SmallIndentation}
     


% ==============================================================
% =================          PROBLEMA 3       ==================
% ==============================================================
\vspace{1em}
\section{3 Problema}

    Dada $A \in M_{m \times n}(\GenericField) \ 0_{m \times n}$. Entonces existe un número finito de
    operaciones elementales que transforman a A en una matriz triangular superior.

    % ======== DEMOSTRACION ========
    \begin{SmallIndentation}[1em]
        \textbf{Demostración}:
        
        Esta es una consecuencia del teorema ya visto en clae:

        Sea $A$ una matriz de $m \times n$ de rango $r$, entonces
        tendremos que $r \leq m$ y $r \leq n$
        y por un número finito de operaciones elementales podemos transformar
        a $A$ en la Matriz D.

        D se ve así, ($0_1, 0_2, 0_3$ son matrices cero):
        \begin{align*}
            \pVector{
                I_{r \times r} & 0_1 \\
                0_2            & 0_3 \\
            }
        \end{align*}

        Nota que obviamente esta matriz es triangular superior.
    
    \end{SmallIndentation}



% ==============================================================
% =================          PROBLEMA 4       ==================
% ==============================================================
\clearpage
\section{4 Problema}

    \begin{itemize}
        
        \item 
            $rango\pVector{
                1 & 2  & 0 & 1  & 1 \\
                2 & 4  & 1 & 3  & 0 \\
                3 & 6  & 2 & 5  & 1 \\
                4 & -8 & 1 & -3 & 1 \\
                }$

            % ======== DEMOSTRACION ========
            \begin{SmallIndentation}[1em]
                \textbf{Demostración}:
                
                Primero voy a afirmar que $(1, 2, 3, 4), (2, 4, 6, -8), (0, 1, 2, 0), (1, 0, 1, 1)$
                son vectores linealmente independientes

                % ======== DEMOSTRACION ========
                \begin{SmallIndentation}[1em]
                    \textbf{Demostración}:
                    
                    Hagamos la clasica, es decir:
                    $a(1, 2, 3, 4) + b(2, 4, 6, -8) + c(0, 1, 2, 0) + d(1, 0, 1, 1) = (0, 0, 0, 0)$
                    Es decir:
                    \begin{itemize}
                        \item $a + 2b + d = 0$
                        \item $2a + 4b + c = 0$
                        \item $3a + 6b + 2c + d = 0$
                        \item $4a - 8b + d = 0$
                    \end{itemize}

                    Sumemos la primera y al cuarta:
                    $8a + 5d = 0$

                    Ahora usemos la segunda y tercera:
                    $a + 2b - d = 0$

                    Ahora usemos la mas nueva y la cuarta:
                    $8a -3b = 0$

                    Ahora despejamos una variable:
                    $-8b = 0$.

                    Por lo tanto $a = 0$, por lo tanto $d = 0$, por lo tanto $c = 0$.

                \end{SmallIndentation}

                Ahora, como tenemos 4 vectores linealmente independientes con 4 entradas, es claro que son base de $\GenericField^4$, por lo
                tanto es el máximo rango que puede tener la matriz con 4 filas, por lo tanto su rango es $4$.


            \end{SmallIndentation}

        \clearpage

        \item 
            $rango\pVector{
                -9 & 10  & 2  & 0 & 1  & -1 \\
                4  & 2   & 7  & 1 & 3  & 0  \\
                4  & 6   & 6  & 5 & 1  & 1  \\
                -3 & 0   & -8 & 1 & -3 & 1  \\
                }$

            % ======== DEMOSTRACION ========
            \begin{SmallIndentation}[1em]
                \textbf{Demostración}:
                
                Primero voy a afirmar que $(9, 4, 4, 3), (10, 2, 6, 0), (2, 7, 6, -8), (1, 3, 1, -3)$
                son vectores linealmente independientes

                % ======== DEMOSTRACION ========
                \begin{SmallIndentation}[1em]
                    \textbf{Demostración}:
                    
                    Hagamos la clasica, es decir:
                    $a(9, 4, 4, 3) + b(10, 2, 6, 0) + c(2, 7, 6, -8) + d(1, 3, 1, -3) = (0, 0, 0, 0)$
                    Es decir:
                    \begin{itemize}
                        \item $9a + 10b + 2c + d = 0$
                        \item $4a + 2b + 7c + 3d = 0$
                        \item $4a + 6b + 6c + d = 0$
                        \item $3a - 8c - 3d = 0$
                    \end{itemize}

                    Sumemos la segunda y al cuarta:
                    $7a + 8b + 13c = 0$

                    Ahora restamos la tercera y la cuarta:
                    $5a + 4b - 4c = 0$

                    Finalmente vamos a hacer tercera y cuarta:
                    $9a + 18b + 10c = 0$

                    Ahora saquemos ecuaciones la primera nueva y la segunda nueva:
                    $-3a + 21c = 0$

                    Y hagamos lo mismo con la segunda y tercera nueva:
                    $45a + 36b - 36c = 0$ y $-18a - 36b - 20c = 0$

                    Es decir $27a - 56c = 0$, por lo tanto $a = 0$, $c = 0$, por lo tanto $b = 0$ y $d = 0$.

                \end{SmallIndentation}

                Ahora, como tenemos 4 vectores linealmente independientes con 4 entradas, es claro que son base de $\GenericField^4$, por lo
                tanto es el máximo rango que puede tener la matriz con 4 filas, por lo tanto su rango es $4$.


            \end{SmallIndentation}


    \end{itemize}




% ==============================================================
% =================          PROBLEMA 5      ==================
% ==============================================================
\clearpage
\section{5 Problema}

    Dada $A \in M_{m \times n}(\GenericField)$, $ran(A) = 0$ si y solo si $A = 0_{m \times n}$

    % ======== DEMOSTRACION ========
    \begin{SmallIndentation}[1em]
        \textbf{Demostración}:
        
        Recuerda que el rango es el número de columnas linealmente independientes, entonces
        es obvio que la matriz cero, es decir, un conjunto de vectores cero, no genera nada, es decir
        tienen una dimensión cero, es decir tiene rango cero.

        Ahora si tiene rango cero, entonces todos sus vectores son cero, porque porque si hubiera uno que no
        fuera cero, entonces su dimensión ya no sería cero, como todos sus vectores son cero, entonces
        cada elemento de la matriz es cero, es decir, es la matriz cero.
    
    \end{SmallIndentation}
        

% ==============================================================
% =================          PROBLEMA 6      ==================
% ==============================================================
\section{6 Problema}

    Dada $A \in M_{m \times n}(\GenericField)$. Si $c$ es un escalar distinto de cero,
    entonces $ran(cA) = ran(A)$

    % ======== DEMOSTRACION ========
    \begin{SmallIndentation}[1em]
        \textbf{Demostración}:
        
        Por definición tenemos que $R(L_A) = R(L_{cA})$
    
        % ======== DEMOSTRACION ========
        \begin{SmallIndentation}[1em]
            \textbf{Demostración}:
            
            Por puras propiedades:
            $R(L_A) = L_A(\GenericField^m) = cLA(\GenericField^m) = L_{cA}(\GenericField) = R(L_{cA})$
        
        \end{SmallIndentation}
            
    \end{SmallIndentation}
        


% ==============================================================
% =================          PROBLEMA 7       ==================
% ==============================================================
\section{7 Problema}

    Dadas $A, B \in M_{m \times n}(\GenericField)$. Entonces $M(A|B) = (MA|MB)$ para toda
    $M \in M_{m \times m}(F)$

    % ======== DEMOSTRACION ========
    \begin{SmallIndentation}[1em]
        \textbf{Demostración}:
        
        Sea $P = M(A | B)$ y sea $Q = (MA | MB)$. 

        Ahora vamos a mostrar que son iguales entrada a entrada.

        Supongamos que $A, B$ tienen $a$ y $b$ columnas respectivamente.

        \begin{multicols}{2}

            Ahora, como estamos hablando de matrices ampliadas, vamos a ver
            por partes, primero, para las columnas de la $1$ a la $a$ tenemos que:

            \begin{align*}
                [P]_{i, j} 
                    &= \sum_{k = 1}^n [M]_{i, k}A_{k, j}   \Space \forall j \in [1, \dots a]    \\
                    &= MA_{i, j}                                                                \\
                    &= Q_{i, j}
            \end{align*}

            Ahora para $j \in [a + 1, \dots, a + b]$ tenemos algo similar:
            \begin{align*}
                [P]_{i, j} 
                    &= \sum_{k = 1}^n [M]_{i, k}B_{k, j}   \Space \forall j \in [1, \dots a]    \\
                    &= MB_{i, j}                                                                \\
                    &= Q_{i, j}
            \end{align*}

        \end{multicols}

            
    
    \end{SmallIndentation}
        

% ==============================================================
% =================          PROBLEMA 8       ==================
% ==============================================================
\clearpage
\section{8 Problema}

    Dadas $A \in M_{m \times 1}(\GenericField) - 0_{m \times 1}$ y $B \in M_{1 \times n}(\GenericField) - 0_{1 \times 0}$.
    Entonces $rango(AB) = 1$

    % ======== DEMOSTRACION ========
    \begin{SmallIndentation}[1em]
        \textbf{Demostración}:
        
        Ok, este esta muy muy bueno.
        Supongamos a $A = \pVector{a_1 \\ a_2 \\ \vdots \\ a_m}$ y $B = \pVector{a_1 & a_2 & \vdots & a_m}$

        Veamos como se ve la matriz $AB$:
        \begin{align*}
            [AB]_{i, j}
                &= \sum_{k = 1}^1 [A]_{i, k} [B]_{k, j}         \\
                &= [A]_{i, 1} [B]_{1, j}                        \\
                &= a_i b_j                                      
        \end{align*}

        Ahora, para encontrar el rango de la matriz haremos primero un cambio de columna, una operación que conserva el rango
        y vamos a mover la columna donde $b_k$ sea diferente de cero, nota que como no es puros ceros, tiene que haber mínimo uno
        y ese lo intercambiamos con la columna 1 con la columna $k$.

        Ahora si vamos a calcular el rango de esta nueva matriz.

        Ahora, considera la columna $r$ con $r > 1$, entonces mira que la columna $r$ se puede expresar usando la columna $1$
        pues $\pVector{a_1 b_r \\ a_2 b_r \\ \vdots \\ a_m b_r} = \dfrac{b_r}{b_1} \pVector{a_1 b_1 \\ a_2 b_1 \\ \vdots \\ a_m b_1}$

        Por lo tanto solo tenemos una columna linealmente independiente.
    
    \end{SmallIndentation}
        



% ==============================================================
% =================          PROBLEMA 9       ==================
% ==============================================================
\clearpage
\section{9 Problema}

    Dada $A \in M_{m \times n}(\GenericField)$, tal que $ran(A) = m$. Entonces existe 
    $B \in M_{n \times m}(\GenericField)$ y $AB = Id_m$.

    % ======== DEMOSTRACION ========
    \begin{SmallIndentation}[1em]
        \textbf{Demostración}:
        

        Sea $\beta$ la base estandar para $\GenericField^m$, es decir
        $\beta = \Set{e_1, e_2, e_3, \dots, e_m}$.

        Ya que el tango de $A = m$, entonces sabemos que $L_A$ es una función suprayectiva,
        es decir podemos encontrar un vector $\vec x_i$ en $\GenericField^m$ tal que $L_A(\vec x_i) = e_i$
        para cualquier $i$ de $1$ a $m$.

        Por lo tanto esa matriz, la llamemos $B$, aquella de la i-ésima columna sea $\vec x_i$.

        Es decir, será una matriz de $n \times m$, y ya que $A \vec x_i = e_i$ tendremos que $AB = Id_m$
    
    \end{SmallIndentation}







\end{document}